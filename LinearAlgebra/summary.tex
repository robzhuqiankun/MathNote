% !TeX spellcheck = en_US
\documentclass[letterpaper]{article}

\usepackage{times}
\usepackage{graphicx}
\usepackage{balance}  % for  \balance command ON LAST PAGE  (only there!)
\usepackage{amsfonts}
\usepackage{url}
%\usepackage{algorithm}
%\usepackage{algorithmic}
\makeatletter
\newif\if@restonecol
\makeatother
\let\algorithm\relax
\let\endalgorithm\relax
\usepackage[linesnumbered,boxed,ruled,vlined]{algorithm2e}
\usepackage{float}
\usepackage{amssymb,amsmath,epsfig,graphics,enumerate,float,subfigure}
\usepackage{color}
\usepackage{xspace}
\usepackage{multirow}




\def\ie{$i.e.$}
\def\prox{\textrm{prox}}
\def\dom{\textrm{dom}}


\begin{document}
	
\section{definition}
\paragraph{F means R or C}
\paragraph{Direct Sum $\oplus$}:\\
$V=U_1\oplus U_2$, if each element of $V$ can be written as $u_1+u_2$ where each $u_i \in U_i$.
\paragraph{Null Spaces and Ranges}:\\
\begin{displaymath}
null\ T = \{v \in V : Tv = 0\}
\end{displaymath}
\begin{displaymath}
range\ T = \{Tv  : v \in V\}
\end{displaymath}
\paragraph{Injective}:\\
$T: V \to W$ is a injective linear map,
\begin{displaymath}
\textrm{If whenever }u,v \in V and\  Tu=Tv, \textrm{We have }u=v.
\end{displaymath}
or equally,
\begin{displaymath}
null\ T =\{0\}
\end{displaymath}
\paragraph{Surjective}:\\
$T: V \to W$ is a surjective linear map, if its range equals $W$
\paragraph{theorem}:
dim $V$ = dim null $T$ + dim range $T$.
\paragraph{theorem}:
$T: V \to W$ is invertible $\iff$ it is injective and surjective. In this case, $V$ and $W$ are called isomorphic with same dim.
\paragraph{theorem}:
If $T: V \to V$ (is a operator). $T$ is invertible $\iff$ $T$ is injective $\iff$ $T$ is surjective. 
\paragraph{Invariant Subspaces}:
\begin{displaymath}
V=U_1\oplus...\oplus U_m,
\end{displaymath}
$T\arrowvert_{U_j}$ denotes restrict $T \in L(V)$ on $U_j$. $U$ is invariant if $T\arrowvert_{U}$ is an operator on $U$.

\paragraph{Eigenvalue and Eigenvector}:
Suppose $U$ is a 1-dim subspace of V,
\begin{displaymath}
U=\{au:a \in F\}
\end{displaymath}
If $T$ is invariant on $U$,
\begin{displaymath}
Tu=\lambda u,
\end{displaymath}
$\lambda$ is an eigenvalue and $u$ is an eigenvector. The set of eigenvectors $U$ equals null$(T-\lambda I)$.
\paragraph{theorem}: Eigenvectors for distinct eigenvalues are linear independent.


\paragraph{Upper Triangular Matrix}
\begin{enumerate}
	\item
	span$(v1,v2...vk)$ is invariant under $T$ for each $k = 1,2...n$.
	\item
	$T$ is invertible $\iff$ all entries on the diagonal are nonzero which are eigenvalues.
\end{enumerate}

\paragraph{adjoint}
Let $T: V \to W$, the adjoint of $T$ denoted as $T^*$. $T^*w$ is the unique vector such that
\begin{displaymath}
<Tv,w>\ =\ <v,T^*w>
\end{displaymath}
The matrix of $T^*$ is the conjugate transpose of the matrix of $T$.\\
$T$ is self-adjoint if $T=T^*$.

\paragraph{normal operator}
$T \in L(V)$ is normal if $TT^*=T^*T$.\\
equally $\|Tv\|=\|T^*v\|$
\paragraph{real spectral theorem}
$T=T* \iff V$ has an orthogonal basis consisting of eigenvectors of $T$   

\paragraph{normal matrix}:\\
2-dim real matrix is of the form
$\begin{bmatrix}
 a & -b \\
 b & a 
 \end{bmatrix}$\\
n-dim real normal matrix is a block diagonal matrix of the form
$\begin{bmatrix}
A_1 && 0 \\
&...&\\
0 &&  A_n
\end{bmatrix}$where each block is 1-dim matrix or $2\times2$ matrix of the form
$\begin{bmatrix}
a & -b \\
b & a 
\end{bmatrix}$\quad with $b>0$.

\paragraph{Positive semidefinite operator}:\\
$T \in L(V)$ is called positive if $T$ is self-adjoint and 
\begin{displaymath}
\langle Tv,v \rangle \geq 0
\end{displaymath}

\paragraph{theorem}: Every p.s.d. matrix has an unique p.s.d. square root.

\paragraph{theorem}: For any $T \in L(V)$, $TT^*$ and $T^*T$ are p.s.d. That is obvious for $\mathbb{R}$, Since $AA^\top$ is p.s.d. as $vAA^\top v^\top = \langle vA,vA \rangle \geq 0$ 

\paragraph{isometry}:\\
$S \in L(V)$ is isometry if
\begin{displaymath}
\|Sv\|=\|v\|
\end{displaymath}for all $v \in V$. We have

\begin{displaymath}
SS^*=I=S^*S
\end{displaymath}
Obviously, $S$ is normal.

\paragraph{polar decomposition}:\\
$T \in L(V)$, we have 
\begin{displaymath}
T=S\sqrt{T^*T}
\end{displaymath}
with $S$ is isometry.

\paragraph{singular values}: Eigenvalues of $\sqrt{T^*T}$.

\paragraph{trace}:\\
Let $A$ is a $n\times n$ matrix.
\begin{displaymath}
trace(A)=\sum_i{a_{ii}}=\sum_i{\lambda_i}
\end{displaymath}
\paragraph{determinant}:\\
Let $A$ is a $n\times n$ matrix.
\begin{displaymath}
det(A)=\prod_i{\lambda_i}
\end{displaymath}
\end{document}
